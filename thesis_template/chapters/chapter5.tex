\chapter{Evaluation and Discussion}
\label{chap:eval}
\begin{longtable}{c|c}
\caption[This is the title I want to appear in the List of Tables]{Simulation Parameters} \label{table:fifsimulation_params} \\
\hline
A & B  \\
\hline
\endfirsthead
\multicolumn{2}{@{}l}{} \\
\hline
A & B \\
\hline
\endhead
\hline
 \textbf{Parameter} & \textbf{Value}\\
 \hline
 Number of vehicles & $|\mathcal{V}|$\\
 \hline
 Number of RSUs & $|\mathcal{U}|$\\
 \hline
 RSU coverage radius & 150 m\\
 \hline
 V2V communication radius & 30 m\\
 \hline
 Smart vehicle antenna height & 1.5 m\\
 \hline
 RSU antenna height & 25 m\\
 \hline
 Smart vehicle maximum speed & $v_{max}$ m/s\\
 \hline
 Smart vehicle minimum speed & $v_{min}$ m/s\\
 \hline
 Common smart vehicle cache capacities & $[50, 100, 150, 200, 250]$ mb\\
 \hline
 Common RSU cache capacities & $[5000,1000,1500,2000,2500]$ mb\\
 \hline
 Common backhaul rates & $[75, 100, 150]$ mb/s\\
 \hline
\end{longtable}

\begin{figure}[hbt]
\centering
\includegraphics[]{figs/inno.png}
\caption{One kernel at $x_s$ (\emph{dotted kernel}) or two kernels at
$x_i$ and $x_j$ (\textit{left and right}) lead to the same summed estimate
at $x_s$. This shows a figure consisting of different types of
lines. Elements of the figure described in the caption should be set in
italics, in parentheses, as shown in this sample caption.}
\label{fig:fifex}
\end{figure}

\ldots
