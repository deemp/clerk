\chapter{Literature Review}
\label{chap:lr}
\chaptermark{Second Chapter Heading}

Objective: Design and implement a type-safe eDSL (embedded domain specific language) and a framework for spreadsheet generation in Haskell.

\section{Chapter overview}
This chapter presents the literature review process and a survey of previous works on eDSL design and functional programming approaches to spreadsheets generation. \Cref{sec:research_questions} lists the literature review questions. Next, \cref{sec:searchProcess} summarizes the initial literature search process and its results. \Cref{sec:relevance} presents a set of relevance criteria. Finally, \cref{sec:selectedLiterature} overviews the answers to the literature review questions presented in \cref{sec:research_questions}.

\section{Literature review questions} \label {sec:research_questions}
The literature review aims at answering the following questions:
% \begin{itemize}[noitemsep]
% \end{itemize}
\setlist{nolistsep}
\begin{enumerate}[noitemsep]
    \item What were the previous attempts at type-safe spreadsheet generation via Haskell?
    \item Which Haskell features were used in the existing eDSLs?
    \item Which DSL design techniques were used in such eDSLs?
\end{enumerate}

\section{Search engines and queries} \label{sec:searchProcess}
To begin with, the following search engines were used: Semantic Scholar \cite{noauthor_semantic_nodate}, Hackage \cite{noauthor_packages_nodate}, HaskellWiki \cite{noauthor_haskellwiki_nodate}, Google \cite{noauthor_google_nodate-1}, and GitHub \cite {noauthor_build_nodate}. These platforms were selected as all of them provided short annotations of stored resources. Later on, such annotations accelerated the preliminary selection of relevant articles and projects. Additionally, different sets of queries were used on each platform depending on:
\begin{enumerate*}[ label=\arabic*) ]
    \item a platform's search mechanism;
    \item numbers of on-topic results obtained via other queries on that platform
\end{enumerate*}.
\Cref{table:search} demonstrates the search engines, queries, and the numbers of preliminary selected search results, excluding duplicates.

\newcommand{\centeredHeader}[1]{\multicolumn{1}{|c|}{\textbf{#1}}}

% https://texblog.org/2011/05/15/multi-page-tables-using-longtable/

\begin{longtable}{|l|l|r|}
    \caption[]{Search results} \label{table:search}                                                       \\
    \hline
    \centeredHeader{Search engine} & \centeredHeader{Search queries}           & \centeredHeader{Results} \\
    \hline
    \endfirsthead
    \hline
    \centeredHeader{Search engine} & \centeredHeader{Search queries}           & \centeredHeader{Results} \\
    \hline
    \endhead
    Semantic Scholar
                                   & haskell embedded domain specific language & 25                       \\
                                   & spreadsheet functional programming        & 11                       \\
                                   & haskell edsl                              & 7                        \\
                                   & spreadsheet generation dsl                & 6                        \\
                                   & spreadsheet functional language           & 2                        \\
                                   & functional excel                          & 2                        \\
    \hline
    Google Scholar
                                   & spreadsheet generation                    & 3                        \\
                                   & functional excel                          & 2                        \\
                                   & spreadsheet dsl                           & 2                        \\
                                   & spreadsheet functional programming        & 1                        \\
    \hline
    Hackage
                                   & languages                                 & 755                      \\
                                   & sheet                                     & 4                        \\
    \hline
    HaskellWiki
                                   & Research papers/Domain specific languages
                                   & 48                                                                   \\
                                   & Embedded domain specific language         &
    6                                                                                                     \\
    \hline
    GitHub
                                   & excel language:Haskell                    & 7                        \\
                                   & spreadsheet language:Haskell              & 7                        \\
    \hline
    YouTube
                                   & Lambdaconf DSL                            & 2                        \\
    \hline
\end{longtable}

\section{Relevance criteria} \label {sec:relevance}

It was decided that each relevant work should:

\begin{enumerate}[noitemsep, label=\arabic*) ]
    \item Be published 1999 or later. A significant number of papers on eDSLs were published between the publications of \textit{Haskell 98} and \textit{Haskell 2010} standards;
    \item Be written in English;
    \item Show Haskell implementation source code or contain a link to such code;
    \item Desirably, explain how DSL design techniques were implemented in Haskell.
    \item Desirably, demonstrate a way to model or generate spreadsheets via Haskell;
\end{enumerate}

\section{Selected literature overview} \label{sec:selectedLiterature}

\subsection{Works on spreadsheet generation}

\subsection{Haskell features in existing eDSLs}

\begin{longtable}{|l|l|}
    \caption[]{Haskell features in existing eDSLs} \label{table:features}                                                                                                                     \\
    \hline
    \centeredHeader{Feature}       & \centeredHeader{eDSL paper}                                                                                                                              \\
    \hline
    \endfirsthead
    \hline
    \centeredHeader{Search engine} & \centeredHeader{Search queries}                                                                                                                          \\
    \hline
    \endhead
    Template Haskell
                                   & \cite{bernauer_eiger_2022}, \cite{garcia-garland_attribute_2019}, \cite{bedo_bioshake_2019}, \cite{viera_staged_2018}, \cite{grebe_rewriting_2017}       \\
    \hline
    Monads
                                   & \cite{bernauer_eiger_2022}, \cite{valliappan_towards_2020}, \cite{viera_staged_2018}, \cite{ekblad_high-performance_2016}, \cite{thiemann_embedded_2005} \\
    \hline
    Type families
                                   & \cite{evans_circuitflow_2021}                                                                                                                           \\
    \hline
    Custom operators
                                   & \cite{mizzi_artagnan_2018}                                                                                                                \\
    \hline
\end{longtable}

\subsection{DSL design techniques}