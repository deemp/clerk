\chapter{Literature Review}
\label{chap:lr}
\chaptermark{Second Chapter Heading}

Objective: Design and implement a type-safe eDSL (embedded domain specific language) and a framework to declaratively generate spreadsheets in Haskell.

\section{Chapter overview}
This chapter presents the literature review process and surveys previous works on eDSL design and functional programming approaches to spreadsheets generation. \Cref{sec:research_questions} lists the literature review questions. Next, \Cref{sec:searchProcess} summarizes the initial literature search process and its results. \Cref{sec:relevance} presents a set of relevance criteria. Finally, \Cref{sec:selectedLiterature} overviews the answers to the literature review questions from \Cref{sec:research_questions}.

\section{Literature review questions} \label {sec:research_questions}
The literature review aims at answering the following questions:
% \begin{itemize}[noitemsep]
% \end{itemize}
\setlist{nolistsep}
\begin{enumerate}[noitemsep]
    \item What were the previous attempts at type-safe spreadsheet generation via Haskell?
    \item Which Haskell features were used in the existing eDSLs?
    \item Which DSL design techniques were used in such eDSLs?
\end{enumerate}

\section{Search engines and queries} \label{sec:searchProcess}
To begin with, the following search engines were used: Semantic Scholar \cite{noauthor_semantic_nodate}, Hackage \cite{noauthor_packages_nodate}, HaskellWiki \cite{noauthor_haskellwiki_nodate}, Google \cite{noauthor_google_nodate-1}, and GitHub \cite {noauthor_build_nodate}. These platforms were selected as all of them provided short annotations of stored resources. Later on, such annotations accelerated the preliminary selection of relevant articles and projects. Additionally, different sets of queries were used on each platform depending on the following:
\begin{enumerate*}[ label=\arabic*) ]
    \item a platform's search mechanism;
    \item numbers of on-topic results obtained via other queries on that platform
\end{enumerate*}.
\Cref{table:search} demonstrates the search engines, queries, and the numbers of preliminary selected search results, excluding duplicates.

\newcommand{\centeredHeader}[1]{\multicolumn{1}{|c|}{\textbf{#1}}}

% https://texblog.org/2011/05/15/multi-page-tables-using-longtable/

\begin{longtable}{|l|l|r|}
    \caption[]{Search results} \label{table:search}                                                       \\
    \hline
    \centeredHeader{Search engine} & \centeredHeader{Search queries}           & \centeredHeader{Results} \\
    \hline
    \endfirsthead
    \hline
    \centeredHeader{Search engine} & \centeredHeader{Search queries}           & \centeredHeader{Results} \\
    \hline
    \endhead
    Semantic Scholar
                                   & haskell embedded domain specific language & 25                       \\
                                   & spreadsheet functional programming        & 11                       \\
                                   & haskell edsl                              & 7                        \\
                                   & spreadsheet generation dsl                & 6                        \\
                                   & spreadsheet functional language           & 2                        \\
                                   & functional excel                          & 2                        \\
    \hline
    Google Scholar
                                   & spreadsheet generation                    & 3                        \\
                                   & functional excel                          & 2                        \\
                                   & spreadsheet dsl                           & 2                        \\
                                   & spreadsheet functional programming        & 1                        \\
    \hline
    Hackage
                                   & languages                                 & 755                      \\
                                   & sheet                                     & 4                        \\
    \hline
    HaskellWiki
                                   & Research papers/Domain specific languages
                                   & 48                                                                   \\
                                   & Embedded domain specific language         &
    6                                                                                                     \\
    \hline
    GitHub
                                   & excel language:Haskell                    & 7                        \\
                                   & spreadsheet language:Haskell              & 7                        \\
    \hline
    YouTube
                                   & Lambdaconf DSL                            & 2                        \\
    \hline
\end{longtable}

\section{Relevance criteria} \label {sec:relevance}

Each relevant work should be or do as follows:

\begin{enumerate}[noitemsep, label=\arabic*) ]
    \item Be published 1999 or later. A significant number of papers on eDSLs were published between the publications of \textit{Haskell 98} and \textit{Haskell 2010} standards;
    \item Be written in English;
    \item Show Haskell implementation source code or contain a link to such code;
    \item Desirably, explain how DSL design techniques were implemented in Haskell.
    \item Desirably, demonstrate a way to model or generate spreadsheets via Haskell;
\end{enumerate}

\section{Selected literature overview} \label{sec:selectedLiterature}

\subsection{Works on spreadsheet generation}

Several works discussed spreadsheet generation via Haskell.
Wakeling \cite{wakeling_spreadsheet_2007} made a Haskell backend for Microsoft Excel. His approach was to write Haskell functions in a separate file, enter them as comments into a spreadsheet, and make a Haskell interpreter HUGS convert them into numeric values. The author admitted that the shortcomings of this interface were slowness of calculations and lack of error reporting. Additionally, no formula could be written in a spreadsheet-native expression language. Later, Zaborsky created a Haskell library Xlsx \cite{github_xlsx} for reading, imperative construction, and writing spreadsheets in Office Open XML xlsx format. The weakness of this library is that it did not provide a DSL to express formulas in a form other than plain text. Next, Thomasson \cite{github_hcalc} made a simple DSL for spreadsheets generation. The main drawbacks of his program were that its outputs were not suitable for existing spreadsheet systems. In addition, the author used fixed cell references which made spreadsheet layouts inflexible. One of the later works was the ComonadSheet \cite{github_comonadsheet} library. It featured support for relative and absolute cell references and infinite cell streams. Nevertheless, it shared the problem of fixed cell references with Thomasson's work. Further, Kudasov \cite{youtube_kudasov} suggested the notion of tables and their monadic composition. This approach allows to represent tables and their headers as trees and provide a Haskell eDSL for a spreadsheet description. Although the author did not elaborate on cell reference tracking and formulas expression, his ideas lay the foundation of this thesis.

\subsection{Haskell features in existing eDSLs}

\begin{longtable}{|l|l|}
    \caption[]{Haskell features in existing eDSLs} \label{table:haskellFeatures}                                                                                                        \\
    \hline
    \centeredHeader{Feature}       & \centeredHeader{eDSL paper}                                                                                                                        \\
    \hline
    \endfirsthead
    \hline
    \centeredHeader{Search engine} & \centeredHeader{Search queries}                                                                                                                    \\
    \hline
    \endhead
    Template Haskell
                                   & \cite{bernauer_eiger_2022}, \cite{garcia-garland_attribute_2019}, \cite{bedo_bioshake_2019}, \cite{viera_staged_2018}, \cite{grebe_rewriting_2017} \\
    \hline
    Monads
                                   & \cite{bernauer_eiger_2022}, \cite{viera_staged_2018},
    \cite{valliappan_towards_2020},  \cite{ekblad_high-performance_2016}, \cite{thiemann_embedded_2005}                                                                                 \\
    \hline
    Type families
                                   & \cite{ekblad_high-performance_2016}, \cite{evans_circuitflow_2021}                                                                                 \\
    \hline
    Custom operators
                                   & \cite{mizzi_artagnan_2018}                                                                                                                         \\
    \hline
\end{longtable}

\subsection{DSL design techniques} \label{sec:dslDesignTechniques}

Bernauer and Eisenberg \cite{bernauer_eiger_2022} used runtime instance lookup for data deserialization. Such a lookup feature was enabled by collecting all class instances of a type into a monoid and putting them into a map indexed with type names. The authors also claimed that violating the monad laws was permittable in their case, though a more correct approach would be to use the recent \texttt{ApplicativeDo} GHC extension. Bernauer and Eisenberg argued that this extension would provide users with convenient \texttt{do-notation} and increase the implementation efficiency.

Ekblad \cite{ekblad_high-performance_2016} suggested using closed type families. This approach allows for writing high-level code in a Haskell eDSL and would guarantee the type-safe conversion into an intermediate language. Furthermore, he noted that to make generated code efficient one has to make inefficient constructs irrepresentable in an eDSL. Another possible approach mentioned \cite{ekblad_high-performance_2016} was function inlining. However, this approach may lead to a problem of having too many local variables at once.

Evans \textit{et al.} \cite{evans_circuitflow_2021} mentioned Indexed Monadic Catamorphism, a technique for recursion elimination from foldings of datatypes. To make catamorphisms modular, they further encoded folding algebra as a type class. Furthermore, Evans \textit{et al.} introduced custom operators to make the code more readable.

\section{Conclusion}

Overall, no works were discovered that fully addressed the problem of type-safe declarative spreadsheet generation using Haskell. That is why, this thesis will build upon the work by Kudasov \cite{youtube_kudasov} and apply the DSL design techniques mentioned in \Cref{sec:dslDesignTechniques}.