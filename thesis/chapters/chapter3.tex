% LIMA_DISABLE

% {- FOURMOLU_DISABLE -}
% {-# OPTIONS_GHC -Wno-redundant-constraints #-}
% {-# LANGUAGE OverloadedStrings #-}
% {-# LANGUAGE DerivingVia #-}
% {-# LANGUAGE ScopedTypeVariables #-}
% {-# LANGUAGE ViewPatterns #-}
% {-# LANGUAGE RecordWildCards #-}
% {-# LANGUAGE UndecidableInstances #-}
% {-# LANGUAGE DeriveAnyClass #-}
% {-# LANGUAGE MonoLocalBinds #-}
% {- FOURMOLU_ENABLE -}

% -- | @Clerk@ library
% module Clerk (
%   -- * Coords
%   Coords,
%   mkCoords,
%   ToCoords (..),
%   FromCoords (..),
%   IsCoords,

%   -- * Cell references
%   Ref,
%   row,
%   col,
%   ref,
%   val,

%   -- * Changing types
%   UnsafeChangeType (..),
%   as,

%   -- * Cell formatting
%   InputIndex,
%   FormatCell,
%   CellTemplate,
%   FormattedMap,
%   FMTransform,
%   WSTransform,
%   Transform,
%   FCTransform,
%   horizontalAlignment,
%   mkColor,
%   blank,
%   ToARGB (..),

%   -- * Templates
%   Row,
%   RowI,
%   RowIO,
%   Template,

%   -- * Columns
%   ColumnsProperties,
%   columnWidthFormatRef,
%   columnWidthRef,
%   columnWidth,
%   columnRef,
%   column,

%   -- * Sheet builder
%   Sheet,
%   placeN,
%   place1,
%   place,
%   evalSheetDefault,

%   -- * Expressions
%   Expr,
%   Formula,
%   ToFormula (..),
%   NumOperator,
%   (.+),
%   (.-),
%   (.*),
%   (./),
%   (.:),
%   (.^),
%   (.^^),
%   (.**),
%   (.<),
%   (.>),
%   (.<=),
%   (.>=),
%   (.=),
%   (.<>),
%   (.&),
%   fun,
%   -- TODO work on default types
%   Range,
%   FunName,

%   -- * Cells
%   CellData,
%   ToCellData (..),

%   -- * xlsx
%   composeXlsx,
%   writeXlsx,

%   -- * For examples
%   SheetState (..),
%   RowState,
%   RowShow (..),
%   evalRow,
%   mkRef,
%   showFormula,
% ) where

% import Codec.Xlsx qualified as X
% import Codec.Xlsx.Formatted qualified as X
% import Control.Monad (forM, void, zipWithM)
% import Control.Monad.State (MonadState, StateT (StateT), evalStateT, get, gets, modify)
% import Control.Monad.Trans.Writer (execWriter, runWriter)
% import Control.Monad.Writer (MonadWriter (..), Writer)
% import Data.ByteString.Lazy qualified as LBS
% import Data.Char (isAlpha, ord)
% import Data.Default (Default (..))
% import Data.Foldable (Foldable (..))
% import Data.Kind (Type)
% import Data.Map.Strict qualified as Map (Map, insert)
% import Data.Maybe (isJust, maybeToList)
% import Data.Text (Text)
% import Data.Text qualified as T
% import Data.Time.Clock.POSIX (getPOSIXTime)
% import GHC.Generics (Generic)
% import Lens.Micro (Lens', lens, (%~), (&), (+~), (.~), (?~), (^.))
% import Unsafe.Coerce (unsafeCoerce)

% LIMA_ENABLE

This chapter describes low-level details of the \texttt{clerk} library implementation. First, the used types and their purpose are explained. Following that, the main helper functions are presented. Finally, a simple example is given to demonstrate the capabilities of \texttt{clerk}.

\section{Imports}

Here are the necessary imports.

% SINGLE_LINE FOURMOLU_DISABLE

% SINGLE_LINE FOURMOLU_ENABLE

\section{Types}
\label{sec:types}

The core type of the library is \texttt{Row}. This is a monad for construction of a template for a row of data. It keeps track of the coordinates of the current cell via the \texttt{StateT Coords m a} transformer. It writes the new cells into a template via the \texttt{Writer (Template input output) a}.

\begin{mycode}
newtype RowIO input output a = Row
  {unRow :: StateT RowState (Writer (Template input output)) a}
  deriving newtype (Functor, Applicative, Monad, MonadState RowState, MonadWriter (Template input output))

type RowState = Coords
\end{mycode}

The row templates are applied to a list of inputs, producing a template for a table.
In this table, each cell has an address, data, and formatting. A special type is used to construct formulas to produce data.

\subsection{Addresses}
\label{sec:addresses}

The \texttt{Coords} denote the address of a cell. This data type has a \texttt{Num} instance that provides cell arithmetics with them like shifts along sheet axes in the directions given as another \texttt{Coords}. Additionally, this type has a \texttt{Show} instance that translates it into valid spreadsheet addresses.

\begin{mycode}
data Coords = Coords
  { _col :: Int
  , _row :: Int
  , _coordsWorksheetName :: T.Text
  , _coordsWorkbookPath :: FilePath
  }
  deriving (Generic, Default, Show)
\end{mycode}

Typeclasses \texttt{ToCoords} and \texttt{FromCoords} allow to generalize working with data that is convertible to and from \texttt{Coords}. Based on these typeclasses, a pair of lenses is provided for convenient work with \texttt{Coords}-like data types.

\begin{mycode}
row :: IsCoords a => Lens' a Int
col :: IsCoords a => Lens' a Int
\end{mycode}

Such data types include \texttt{Ref}s, which are addresses of cells plus a phantom type to allow type-safe operations. For example, for a \texttt{Ref Int}, arithmetic operations are only allowed with another \texttt{Ref Int}. \texttt{Ref} inherits the \texttt{Num} instance of \texttt{Coords}.

\begin{mycode}
newtype Ref a = Ref {unRef :: Coords}
  deriving (Num) via Coords
\end{mycode}

The phantom type transformations are made possible via the \texttt{UnsafeChangeType} class.

\begin{mycode}
class UnsafeChangeType (a :: Type -> Type) where
  unsafeChangeType :: forall c b. a b -> a c
\end{mycode}

\subsection{Cell data}
\label{sec:celldata}

When building a template, all \texttt{input}s are translated into \texttt{CellData}, which unites the data types from \texttt{xlsx}.

\begin{mycode}
data CellData
  = CellFormula X.CellFormula
  | CellValue X.CellValue
  | CellComment X.Comment
  | CellEmpty
  deriving (Show)
\end{mycode}

That is why, one can use a type synonym for building row templates.

\begin{mycode}
type RowI input a = RowIO input CellData a
\end{mycode}

There is a typeclass \texttt{ToCellData} that allows to work with arbitrary types convertible to \texttt{CellData}.

\begin{mycode}
class Show a => ToCellData a where
  toCellData :: a -> Row CellData
\end{mycode}

\subsection{Formatting}
\label{sec:formatting}

To store the additional information about a cell, a \texttt{CellTemplate} type is introduced.

\begin{mycode}
data CellTemplate input output = CellTemplate
  { mkOutput :: input -> output
  , fmtCell :: FormatCell
  , columnsProperties :: Maybe X.ColumnsProperties
  }
\end{mycode}

The \texttt{FormatCell} type synonym denotes a function that produces a formatted cell based on that cell's address, index in the input list, and the data.

\begin{mycode}
type FormatCell =
  forall a b.
  (ToCoords a, ToCellData b) =>
  (a -> InputIndex -> b -> Row X.FormattedCell)
\end{mycode}

\subsection{Formulas}
\label{sec:formulas}

The spreadsheet formulas are modeled via recursive data types and have a phantom type to store the resulting type of a formula.

\begin{mycode}
data Expr t
  = EBinaryOp {binOp :: BinaryOperator, arg1 :: Expr t, arg2 :: Expr t}
  | EFunction {fName :: T.Text, fArgs :: [Expr t]}
  | ERef {r :: Ref t}
  | ERange {ref1 :: Ref t, ref2 :: Ref t}
  | EValue {value :: t}
  | EUnaryOp {unaryOp :: UnaryOp, arg :: Expr t}

data BinaryOperator
  = OpAdd
  | OpSubtract
  | OpMultiply
  | OpDivide
  | OpPower
  | OpLT
  | OpGT
  | OpLEQ
  | OpGEQ
  | OpEQ
  | OpNEQ

data UnaryOp
  = OpNeg
\end{mycode}

To introduce the new functionality on top of \texttt{Expr}, the \texttt{Formula} is used.

\begin{mycode}
newtype Formula t = Formula {unFormula :: Expr t}
  deriving newtype (UnsafeChangeType)
\end{mycode}

Formulas are constructed from values and addresses combined via operators and functions.

\subsubsection{Operators}
\label{sec:operators}

A number of operators are used to build formulas.
These operators resemble ones from spreadsheet systems and have the same fixities as similar \texttt{Haskell} operators.

\begin{itemize}
  \item For constructing ranges
  \begin{mycode}
(.:) :: forall (a :: Type) (b :: Type). Ref a -> Ref b -> Formula Range
  \end{mycode}
  \item Arithmetic operators
  \begin{mycode}
type NumOperator a b c d e = (Num a, Num c, ToFormula (d a), ToFormula (e b)) => d a -> e b -> Formula c
(.+) :: NumOperator a a a d e
(.-) :: NumOperator a a a d e
(./) :: (Fractional a) => NumOperator a a a d e
(.*) :: NumOperator a a a d e
(.^) :: (Num a, Integral b) => NumOperator a b a d e
(.^^) :: (Fractional a, Integral b) => NumOperator a b a d e
(.**) :: (Floating a) => NumOperator a a a d e
  \end{mycode}
  \item Operators that produce boolean values
  \begin{mycode}
type BoolOperator a b c = (Ord a, ToFormula (b a), ToFormula (c a)) => b a -> c a -> Formula Bool
(.<) :: BoolOperator a b c
(.>) :: BoolOperator a b c
(.<=) :: BoolOperator a b c
(.>=) :: BoolOperator a b c
(.=) :: BoolOperator a b c
(.<>) :: BoolOperator a b c
  \end{mycode}
\end{itemize}

\subsubsection{Functions}

A user may want to construct custom functions. It is made possible via another typeclass and a helper method. To set the types of arguments, a user can specify the type \texttt{t}.

\begin{mycode}
type FunName = T.Text

class MakeFunction t where
  makeFunction :: FunName -> [Formula s] -> t

fun :: MakeFunction t => FunName -> t
\end{mycode}

% LIMA_DISABLE

% instance Default T.Text where
%   def :: T.Text
%   def = ""

% -- | Make 'Coords' from a column index and a row index
% mkCoords :: Int -> Int -> Sheet Coords
% mkCoords _col _row = do
%   SheetState{_sheetWorkbookPath = _coordsWorkbookPath, _sheetWorksheetName = _coordsWorksheetName} <- get
%   pure Coords{_col, _row, ..}

% -- | Convertible to 'Coords'
% class ToCoords a where
%   toCoords :: a -> Coords

% -- | Convertible from 'Coords'
% class FromCoords a where
%   fromCoords :: Coords -> a

% type IsCoords a = (FromCoords a, ToCoords a)

% instance ToCoords Coords where
%   toCoords :: Coords -> Coords
%   toCoords = id

% instance FromCoords Coords where
%   fromCoords :: Coords -> Coords
%   fromCoords = id

% -- | Show in context of a row
% class Show a => RowShow a where
%   rowShow :: a -> Row T.Text

% instance RowShow Coords where
%   rowShow :: Coords -> Row T.Text
%   rowShow cs = do
%     state <- get
%     let
%       prefix
%         | (cs & _coordsWorkbookPath) /= (state & _coordsWorkbookPath) =
%             "'[" <> T.pack (cs & _coordsWorkbookPath) <> "]" <> (cs & _coordsWorksheetName) <> "'!"
%         | (cs & _coordsWorksheetName) /= (state & _coordsWorksheetName) = (cs & _coordsWorksheetName) <> "!"
%         | otherwise = ""
%     pure $ prefix <> toLetters (cs ^. col) <> T.pack (show (cs ^. row))

% -- instance Show a => RowShow a where
% --   rowShow a =

% -- instance Show Coords where
% --   show :: Coords -> T.Text
% --   show (Coords{..}) = "[Book" <> show _coordsWorkbookId <> "]" <> toLetters _col <> show _row

% instance Num Coords where
%   (+) :: Coords -> Coords -> Coords
%   (+) Coords{_row = r1, _col = c1, ..} Coords{_row = r2, _col = c2} = Coords{_row = r1 + r2, _col = c1 + c2, ..}
%   (*) :: Coords -> Coords -> Coords
%   (*) Coords{_row = r1, _col = c1, ..} Coords{_row = r2, _col = c2} = Coords{_row = r1 * r2, _col = c1 * c2, ..}
%   (-) :: Coords -> Coords -> Coords
%   (-) Coords{_row = r1, _col = c1, ..} Coords{_row = r2, _col = c2} = Coords{_row = r1 - r2, _col = c1 - c2, ..}
%   abs :: Coords -> Coords
%   abs Coords{..} = Coords{_row = abs _row, _col = abs _col, ..}
%   signum :: Coords -> Coords
%   signum Coords{..} = Coords{_row = signum _row, _col = signum _col, ..}

%   -- shouldn't be used in lenses as it sets the default values
%   fromInteger :: Integer -> Coords
%   fromInteger x = def{_row = fromIntegral (abs x), _col = fromIntegral (abs x)}

% -- | Letters that can be used in column indices
% alphabet :: String
% alphabet = ['A' .. 'Z']

% -- | Translate a number into a column letters
% --
% -- @
% -- >>> toLetters <$> [1, 26, 27, 52, 78]
% -- ["A","Z","AA","AZ","BZ"]
% --
% -- @
% toLetters :: Int -> T.Text
% toLetters x = f "" (x - 1)
%  where
%   new :: Int -> T.Text -> T.Text
%   new cur acc = T.pack [alphabet !! (cur `mod` 26)] <> acc
%   f :: T.Text -> Int -> T.Text
%   f acc cur = if cur `div` 26 > 0 then f (new cur acc) (cur `div` 26 - 1) else new cur acc

% -- | Translate a column address into a number
% fromLetters :: T.Text -> Int
% fromLetters (T.unpack -> x)
%   | any (`notElem` alphabet) x = error "Column address contains an invalid character"
%   | otherwise = foldl' (\res c -> res * 26 + (ord c - ord 'A' + 1)) 0 x

% {- FOURMOLU_DISABLE -}
% -- $Ref
% {- FOURMOLU_ENABLE -}

% -- | A typed reference to a cell.
% --
% -- The user is responsible for setting the necessary cell type.
% --
% -- The type prevents operations between cell references with incompatible types.
% --
% -- @
% -- >>>str = undefined :: Ref T.Text
% -- >>>str .+ str
% -- No instance for (Num Text) arising from a use of `.+'
% -- In the expression: str .+ str
% -- In an equation for `it_aezf8': it_aezf8 = str .+ str
% --
% -- @
% -- When necessary, one can UNSAFELY change the cell reference type via 'as'
% --
% -- @
% -- >>>int = undefined :: Ref Int
% -- >>>double = undefined :: Ref Double
% -- >>>as int .+ double
% -- Couldn't match expected type `Coords'
% --             with actual type `Text -> FilePath -> Coords'
% -- Probable cause: `Coords' is applied to too few arguments
% -- In the first argument of `Ref', namely `(Coords 1 1)'
% -- In the expression: Ref (Coords 1 1) :: Ref Int
% -- In an equation for `int': int = Ref (Coords 1 1) :: Ref Int
% --
% -- @
% instance ToCoords (Ref a) where
%   toCoords :: Ref a -> Coords
%   toCoords = unRef

% instance FromCoords (Ref a) where
%   fromCoords :: Coords -> Ref a
%   fromCoords = Ref

% -- | A lens for @row@s
% row = lens getter setter
%  where
%   getter (toCoords -> Coords{_row}) = _row
%   setter (toCoords -> Coords{_row, _col, ..}) f = fromCoords $ Coords{_row = f, _col, ..}

% -- | A lens for @col@s
% col = lens getter setter
%  where
%   getter (toCoords -> Coords{_col}) = _col
%   setter (toCoords -> Coords{_row, _col, ..}) f = fromCoords $ Coords{_row, _col = f, ..}

% {- FOURMOLU_DISABLE -}
% -- $ChangeTypes
% {- FOURMOLU_ENABLE -}

% -- | Change the type of something. Use with caution!

% -- | UNSAFELY change the type of something wrapped
% as :: forall c b a. UnsafeChangeType a => a b -> a c
% as = unsafeChangeType

% instance UnsafeChangeType Ref where
%   unsafeChangeType :: Ref b -> Ref c
%   unsafeChangeType (Ref c) = Ref c

% {- FOURMOLU_DISABLE -}
% -- $Formatting
% {- FOURMOLU_ENABLE -}

% -- | Index of an input
% type InputIndex = Int

% -- | Format a single cell depending on its coordinates, index, and data

% -- | Template of a cell with contents, style, column properties

% -- | Map of coordinates to cell formatting
% type FormattedMap = Map.Map (X.RowIndex, X.ColumnIndex) X.FormattedCell

% -- | Transform of a map that maps coordinates to cell formatting
% type FMTransform = FormattedMap -> FormattedMap

% -- | Transform of a worksheet
% type WSTransform = X.Worksheet -> X.Worksheet

% -- | Combined: a transform of a map of formats and a transform of a worksheet
% data Transform = Transform {fmTransform :: FMTransform, wsTransform :: WSTransform}

% instance Semigroup Transform where
%   (<>) :: Transform -> Transform -> Transform
%   (Transform a1 b1) <> (Transform a2 b2) = Transform (a2 . a1) (b2 . b1)

% instance Monoid Transform where
%   mempty :: Transform
%   mempty = Transform id id

% instance Default Transform where
%   def :: Transform
%   def = mempty

% -- | something that can be turned into ARGB
% class ToARGB a where
%   toARGB :: a -> String

% -- | Make a 'FormatCell' for a single color
% --
% -- @show@ on the input should translate into an @ARGB@ color. See 'XS.Color'
% mkColor :: ToARGB a => a -> FormatCell
% mkColor color _ _ c = do
%   cd <- toCellData c
%   pure $
%     X.def
%       & X.formattedCell .~ dataCell cd
%       & X.formattedFormat
%         .~ ( X.def
%               & X.formatFill
%                 ?~ ( X.def
%                       & X.fillPattern
%                         ?~ ( X.def
%                               & ( X.fillPatternFgColor
%                                     ?~ (X.def & X.colorARGB ?~ T.pack (toARGB color))
%                                 )
%                               & ( X.fillPatternType
%                                     ?~ X.PatternTypeSolid
%                                 )
%                            )
%                    )
%            )

% -- | A 'FormatCell' that produces a cell with the given data
% blank :: FormatCell
% blank _ _ cd_ = do
%   cd <- toCellData cd_
%   pure $ X.def & X.formattedCell .~ dataCell cd

% -- | Transform of a formatted cell
% type FCTransform = X.FormattedCell -> X.FormattedCell

% -- | Apply 'FCTransform' to a 'FormatCell' to get a new 'FormatCell'
% (.&) :: FormatCell -> FCTransform -> FormatCell
% fc .& ft = \coords_ index cd -> ft <$> fc coords_ index cd

% infixl 5 .&

% -- | Get a 'FCTransform' with a given horizontal alignment in a cell
% horizontalAlignment :: X.CellHorizontalAlignment -> FCTransform
% horizontalAlignment alignment fc =
%   fc
%     & X.formattedFormat
%       %~ X.formatAlignment
%       ?~ (X.def & X.alignmentHorizontal ?~ alignment)

% {- FOURMOLU_DISABLE -}
% -- $Templates
% {- FOURMOLU_ENABLE -}

% -- | Template for multiple cells
% newtype Template input output = Template [CellTemplate input output]
%   deriving newtype (Semigroup, Monoid)

% -- | Row with a default @()@ input
% type RowO output a = RowIO () output a

% -- | Row with a default @()@ input and a default 'CellData' output
% type Row a = RowO CellData a

% -- | Run builder on given coordinates. Get a result and a template
% runBuilder :: RowIO input output a -> RowState -> (a, Template input output)
% runBuilder builder coord = runWriter (evalStateT (unRow builder) coord)

% -- | Run builder on given coordinates. Get a template
% execRow :: RowIO input output a -> RowState -> Template input output
% execRow builder state = snd $ runBuilder builder state

% -- | Run builder on given coordinates. Get a result
% evalRow :: RowIO input output a -> RowState -> a
% evalRow builder state = fst $ runBuilder builder state

% type RenderTemplate m input output = (Monad m, ToCellData output) => RowState -> InputIndex -> input -> Template input output -> Sheet Transform
% type RenderInputs m input output a = (Monad m, ToCellData output) => [input] -> RowIO input output a -> Sheet (Transform, a)

% -- | Render inputs starting at given coords and using a row. Return the result calculated using the topmost row
% renderInputs :: ToCellData output => RowState -> RenderTemplate Sheet input output -> RenderInputs Sheet input output a
% renderInputs state render inputs row_ = do
%   let
%     ts =
%       [ (newState, template)
%       | inputRow <- [0 .. length inputs - 1]
%       , let newState = state & row +~ inputRow
%             template = execRow row_ newState
%       ]
%     -- result obtained from the top row
%     rowResult = evalRow row_ state
%     transform =
%       fold
%         <$> sequenceA
%           ( zipWith3
%               ( \input inputIndex (st, template) ->
%                   render st inputIndex input template
%               )
%               inputs
%               [0 ..]
%               ts
%           )
%   (,rowResult) <$> transform

% instance FromCoords (X.RowIndex, X.ColumnIndex) where
%   fromCoords :: Coords -> (X.RowIndex, X.ColumnIndex)
%   fromCoords Coords{..} = (fromIntegral _row, fromIntegral _col)

% -- | Render a template with a given offset, input index and input
% renderTemplate :: RenderTemplate Sheet input output
% renderTemplate state inputIndex input (Template columns) = do
%   ps <-
%     zipWithM
%       ( \columnIndex cellTemplate -> do
%           let
%             CellTemplate{..} = cellTemplate
%             leftCell = state
%             cellCol = leftCell ^. col + columnIndex
%           cellCoords <- mkCoords (leftCell ^. row) cellCol
%           let cellData = fst $ runWriter $ flip evalStateT cellCoords $ unRow $ toCellData (mkOutput input)
%               cell = fst $ runWriter $ flip evalStateT cellCoords $ unRow $ fmtCell cellCoords inputIndex cellData
%           let wsTransform
%                 -- add column width only once
%                 -- new properties precede old properties
%                 | inputIndex == 0 = X.wsColumnsProperties %~ (maybeToList columnsProperties ++)
%                 | otherwise = id
%           newCoords <- mkCoords cellCol (leftCell ^. row)
%           let fmTransform = Map.insert (fromCoords newCoords) cell
%           pure def{fmTransform, wsTransform}
%       )
%       [0 ..]
%       columns
%   pure $ fold ps

% {- FOURMOLU_DISABLE -}
% -- $Columns
% {- FOURMOLU_ENABLE -}

% -- | Properties of a column
% newtype ColumnsProperties = ColumnsProperties {unColumnsProperties :: X.ColumnsProperties}

% instance Default ColumnsProperties where
%   def :: ColumnsProperties
%   def =
%     ColumnsProperties
%       X.ColumnsProperties
%         { cpMin = 1
%         , cpMax = 1
%         , cpWidth = Nothing
%         , cpStyle = Nothing
%         , cpHidden = False
%         , cpCollapsed = False
%         , cpBestFit = False
%         }

% -- | A column with a maybe given width and a given cell format. Return a cell reference
% columnWidthFormatRef :: forall a input output. Maybe Double -> FormatCell -> (input -> output) -> RowIO input output (Ref a)
% columnWidthFormatRef width fmtCell mkOutput = do
%   state <- get
%   let columnsProperties =
%         Just $
%           (unColumnsProperties def)
%             { X.cpMin = state ^. col
%             , X.cpMax = state ^. col
%             , X.cpWidth = width
%             }
%   tell (Template [CellTemplate{fmtCell, mkOutput, columnsProperties}])
%   cell <- gets Ref
%   modify (col +~ 1)
%   pure cell

% -- | A column with a given width and cell format. Returns a cell reference
% columnWidthRef :: ToCellData output => Double -> FormatCell -> (input -> output) -> RowI input (Ref a)
% columnWidthRef width fmtCell mkOutput = do
%   state <- get
%   columnWidthFormatRef (Just width) fmtCell (fst . runWriter . flip evalStateT state . unRow . toCellData . mkOutput)

% -- | A column with a given width and cell format
% columnWidth :: ToCellData output => Double -> FormatCell -> (input -> output) -> RowI input ()
% columnWidth width fmtCell mkOutput = void (columnWidthRef width fmtCell mkOutput)

% -- | A column with a given cell format. Returns a cell reference
% columnRef :: ToCellData output => FormatCell -> (input -> output) -> RowI input (Ref a)
% columnRef fmtCell mkOutput = do
%   state <- get
%   columnWidthFormatRef Nothing fmtCell (fst . runWriter . flip evalStateT state . unRow . toCellData . mkOutput)

% -- | A column with a given cell format
% column :: ToCellData output => FormatCell -> (input -> output) -> RowI input ()
% column fmtCell mkOutput = void (columnRef fmtCell mkOutput)

% {- FOURMOLU_DISABLE -}
% -- $Sheet
% {- FOURMOLU_ENABLE -}

% data SheetState = SheetState
%   { _sheetWorksheetName :: T.Text
%   , _sheetWorkbookPath :: FilePath
%   }

% -- | A builder to compose the results of 'Transform's
% newtype Sheet a = Sheet {unSheet :: StateT SheetState (Writer Transform) a}
%   deriving newtype (Functor, Applicative, Monad, MonadWriter Transform, MonadState SheetState)

% -- | Evaluate the result of a sheet with a default state
% evalSheetDefault :: Sheet a -> a
% evalSheetDefault s = fst $ runWriter $ flip evalStateT (SheetState{_sheetWorksheetName = "worksheet", _sheetWorkbookPath = "workbook"}) $ unSheet s

% -- | Starting at a given coordinate, place a list of inputs according to a row builder and return a result
% placeN :: (ToCellData output, ToCoords c) => c -> [input] -> RowIO input output a -> Sheet a
% placeN (toCoords -> state) inputs b = do
%   transformResult <- renderInputs state renderTemplate inputs b
%   tell (fst transformResult)
%   pure (snd transformResult)

% -- | Starting at a given coordinate, place one input according to a row builder and return a result
% place1 :: (ToCellData output, ToCoords c) => c -> input -> RowIO input output a -> Sheet a
% place1 coords_ input = placeN coords_ [input]

% -- | Starting at a given coordinate, place a row builder and return a result
% place :: (ToCellData output, ToCoords c) => c -> RowO output a -> Sheet a
% place coords_ = place1 coords_ ()

% {- FOURMOLU_DISABLE -}
% -- $Expressions
% {- FOURMOLU_ENABLE -}

% -- | Expressions

% -- | Formula

% -- | Something that can be turned into a formula
% class ToFormula a where
%   toFormula :: a -> Formula t

% instance ToFormula (Ref a) where
%   toFormula :: Ref a -> Formula t
%   toFormula (Ref c) = Formula $ ERef (Ref c)

% -- | Convert a reference to a formula
% ref :: Ref a -> Formula a
% ref = toFormula

% instance ToFormula Coords where
%   toFormula :: Coords -> Formula t
%   toFormula c = Formula $ ERef (Ref c)

% instance ToFormula (Expr a) where
%   toFormula :: Expr a -> Formula b
%   toFormula = Formula . unsafeChangeType

% instance ToFormula (Formula a) where
%   toFormula :: Formula a -> Formula b
%   toFormula (Formula f) = Formula $ unsafeChangeType f

% -- TODO dangerous?
% instance {-# OVERLAPPABLE #-} Show a => ToFormula a where
%   toFormula :: a -> Formula b
%   toFormula = Formula . EValue . unsafeCoerce

% showOp2 :: (Show a, RowShow a, Show b, RowShow b) => a -> b -> T.Text -> Row T.Text
% showOp2 c1 c2 operator = do
%   d1 <- rowShow c1
%   d2 <- rowShow c2
%   pure $ d1 <> operator <> d2

% mkOp2 :: (ToFormula a, ToFormula b) => BinaryOperator -> a -> b -> Formula t
% mkOp2 f c1 c2 = Formula $ EBinaryOp f (unFormula $ toFormula c1) (unFormula $ toFormula c2)

% mkNumOp2 :: (Num t, ToFormula a, ToFormula b) => BinaryOperator -> a -> b -> Formula t
% mkNumOp2 = mkOp2

% data Range

% -- | Construct a range expression
% (.:) a b = Formula $ ERange (unsafeChangeType a) (unsafeChangeType b)

% infixr 5 .:

% -- | Convert a value to a formula
% val :: Show a => a -> Formula a
% val a = Formula $ EValue a

% -- | A type for numeric operators

% -- | Construct an addition expression like @A1 + B1@
% (.+) = mkNumOp2 OpAdd

% infixl 6 .+

% -- | Construct a subtraction expression like @A1 - B1@
% (.-) = mkNumOp2 OpSubtract

% infixl 6 .-

% -- | Construct a division expression like @A1 / B1@
% (./) = mkNumOp2 OpDivide

% infixl 7 ./

% -- | Construct a multiplication expression like @A1 * B1@
% (.*) = mkNumOp2 OpMultiply

% infixl 6 .*

% -- | Construct an exponentiation expression like @A1 ^ B1@
% (.^) = mkNumOp2 OpPower

% infixr 8 .^

% -- | Construct an exponentiation expression like @A1 ^ B1@ with 'Fractional' base
% (.^^) = mkNumOp2 OpPower

% infixr 8 .^^

% -- | Construct an exponentiation expression like @A1 ^ B1@ with 'Floating' base
% (.**) = mkNumOp2 OpPower

% infixr 8 .**

% mkBoolOp2 :: (Ord a, ToFormula (b a), ToFormula (c a)) => BinaryOperator -> b a -> c a -> Formula Bool
% mkBoolOp2 f c1 c2 = Formula $ EBinaryOp f (unFormula $ toFormula c1) (unFormula $ toFormula c2)

% -- | Construct a @less-than@ expression like @A1 < B1@
% (.<) = mkBoolOp2 OpLT

% infix 4 .<

% -- | Construct a @greater-than@ expression like @A1 > B1@
% (.>) = mkBoolOp2 OpGT

% infix 4 .>

% -- | Construct a @less-than-or-equal-to@ expression like @A1 <= B1@
% (.<=) = mkBoolOp2 OpLEQ

% infix 4 .<=

% -- | Construct a @greater-than-or-equal-to@ expression like @A1 <= B1@
% (.>=) = mkBoolOp2 OpGEQ

% infix 4 .>=

% -- | Construct a @equal-to@ expression like @A1 = B1@
% (.=) = mkBoolOp2 OpEQ

% infix 4 .=

% -- | Construct a @not-equal-to@ expression like @A1 <> B1@
% (.<>) = mkBoolOp2 OpNEQ

% infix 4 .<>

% instance Show t => Show (Expr t) where
%   show :: Show t => Expr t -> String
%   show (EValue v) = show v
%   show _ = error "Shouldn't be accessed for other constructors"

% instance Show (Expr t) => RowShow (Expr t) where
%   rowShow :: Show (Expr t) => Expr t -> Row T.Text
%   rowShow (EBinaryOp{..}) =
%     showOp2 arg1 arg2 $
%       case binOp of
%         OpAdd -> "+"
%         OpSubtract -> "-"
%         OpMultiply -> "*"
%         OpDivide -> "/"
%         OpPower -> "^"
%         OpLT -> "<"
%         OpGT -> ">"
%         OpLEQ -> "<="
%         OpGEQ -> ">="
%         OpEQ -> "="
%         OpNEQ -> "<>"
%   rowShow (ERef (Ref e)) = rowShow e
%   rowShow (ERange (Ref c1) (Ref c2)) = do
%     d1 <- rowShow c1
%     d2 <- rowShow c2
%     pure $ d1 <> ":" <> d2
%   rowShow (EFunction n args) = do
%     d1 <- forM args rowShow
%     pure $ n <> "(" <> T.intercalate "," d1 <> ")"
%   rowShow (EUnaryOp{..}) =
%     case unaryOp of
%       OpNeg -> rowShow arg
%   rowShow EValue{..} = pure $ T.pack $ show (EValue{..})

% instance UnsafeChangeType Expr where
%   unsafeChangeType :: Expr b -> Expr c
%   unsafeChangeType (EBinaryOp a b c) = EBinaryOp a (unsafeChangeType b) (unsafeChangeType c)
%   unsafeChangeType (ERef (Ref a)) = ERef (Ref a)
%   unsafeChangeType (EFunction n args) = EFunction n (unsafeChangeType <$> args)
%   unsafeChangeType (ERange l r) = ERange (unsafeChangeType l) (unsafeChangeType r)
%   unsafeChangeType (EUnaryOp u v) = EUnaryOp u (unsafeCoerce v)
%   unsafeChangeType (EValue v) = EValue (unsafeCoerce v)

% -- | Name of a function like @SUM@
% instance MakeFunction (Formula a) where
%   makeFunction :: FunName -> [Formula s] -> Formula a
%   makeFunction name args = Formula $ EFunction name (unsafeChangeType . unFormula <$> args)

% instance (Foldable f, MakeFunction t, ToFormula a) => MakeFunction (f a -> t) where
%   makeFunction :: (Foldable f, MakeFunction t, ToFormula a) => FunName -> [Formula s] -> f a -> t
%   makeFunction name args xs =
%     makeFunction
%       name
%       ((unsafeChangeType . toFormula <$> args) ++ foldMap ((: []) . unsafeChangeType . toFormula) xs)

% -- | Construct a function like @SUM(A1,B1)@
% fun n = makeFunction n []

% instance Show (Expr t) => Show (Formula t) where
%   show :: Show (Expr t) => Formula t -> String
%   show (Formula f) = show f

% instance Show (Expr t) => RowShow (Formula t) where
%   rowShow :: Show (Expr t) => Formula t -> Row T.Text
%   rowShow (Formula f) = rowShow f

% {- FOURMOLU_DISABLE -}
% -- $Cells
% {- FOURMOLU_ENABLE -}

% -- | A union of what can be inside a cell
% instance Default CellData where
%   def :: CellData
%   def = CellEmpty

% -- | Convert some Ref component into a cell
% dataCell :: CellData -> X.Cell
% dataCell cd =
%   X.def
%     & case cd of
%       CellValue d -> X.cellValue ?~ d
%       CellFormula d -> X.cellFormula ?~ d
%       CellComment d -> X.cellComment ?~ d
%       CellEmpty -> X.def

% -- | Something that can be turned into 'CellData'
% instance ToCellData T.Text where
%   toCellData :: T.Text -> Row CellData
%   toCellData = pure . CellValue . X.CellText

% instance ToCellData String where
%   toCellData :: String -> Row CellData
%   toCellData = pure . CellValue . X.CellText . T.pack

% instance ToCellData Int where
%   toCellData :: Int -> Row CellData
%   toCellData = pure . CellValue . X.CellDouble . fromIntegral

% instance ToCellData Double where
%   toCellData :: Double -> Row CellData
%   toCellData = pure . CellValue . X.CellDouble

% instance ToCellData Bool where
%   toCellData :: Bool -> Row CellData
%   toCellData = pure . CellValue . X.CellBool

% instance ToCellData CellData where
%   toCellData :: CellData -> Row CellData
%   toCellData = pure

% instance Show (Expr a) => ToCellData (Expr a) where
%   toCellData :: Expr a -> Row CellData
%   toCellData e_ = do
%     e <- rowShow e_
%     pure $
%       CellFormula
%         X.CellFormula
%           { X._cellfAssignsToName = False
%           , X._cellfCalculate = True
%           , X._cellfExpression = X.NormalFormula $ X.Formula e
%           }

% instance (Show (Expr a), Show (Expr a)) => ToCellData (Formula a) where
%   toCellData :: Formula a -> Row CellData
%   toCellData (Formula e) = toCellData e

% {- FOURMOLU_DISABLE -}
% -- $Xlsx
% {- FOURMOLU_ENABLE -}

% -- | Compose an @xlsx@ from a list of sheet names and builders
% composeXlsx :: FilePath -> [(T.Text, Sheet ())] -> X.Xlsx
% composeXlsx path sheetBuilders = workBookWithColumnWidths
%  where
%   getTransform _sheetWorksheetName x =
%     execWriter
%       $ flip
%         evalStateT
%         (SheetState{_sheetWorkbookPath = path, ..})
%       $ unSheet x
%   workBookWithData =
%     flip X.formatWorkbook X.def $
%       (\(name, tf) -> (name, (getTransform name tf & fmTransform) X.def))
%         <$> sheetBuilders
%   workBookWithColumnWidths =
%     workBookWithData
%       & X.xlSheets
%         %~ \sheets ->
%           zipWith
%             ( \x (name, ws) ->
%                 ( name
%                 , ws
%                     & (getTransform name x & wsTransform)
%                     & X.wsColumnsProperties %~ filter (isJust . X.cpWidth)
%                 )
%             )
%             (snd <$> sheetBuilders)
%             sheets

% -- | Lazily write an xlsx
% writeXlsx :: FilePath -> [(T.Text, Sheet ())] -> IO ()
% writeXlsx file sheets = do
%   ct <- getPOSIXTime
%   let xlsx = composeXlsx file sheets
%   LBS.writeFile file $ X.fromXlsx ct xlsx

% {- FOURMOLU_DISABLE -}
% -- $ForExamples
% {- FOURMOLU_ENABLE -}

% mkRef :: String -> Ref a
% mkRef s = fromCoords def{_col, _row}
%  where
%   _col = fromLetters $ T.pack $ takeWhile isAlpha s
%   _row = read $ dropWhile isAlpha s

% showFormula :: RowShow a => a -> Text
% showFormula a = evalRow (rowShow a) def

% LIMA_ENABLE
