\chapter{Evaluation and Discussion}
\label{chap:eval}

This chapter analyzes the research results.

\Cref{eval:findings} presents the main findings that are connected with the research purpose.
\Cref{eval:finding-results} interprets how the research results support these findings.
\Cref{eval:previous-research} contrasts my findings with the results of the past researches.
\Cref{eval:discrepancies} explains the discrepancies between my results and findings.
\Cref{eval:limitations} describes the limitations of my research.
\Cref{eval:unexpected} lists the unexpected findings.
\Cref{eval:outer-applications} suggests the possible applications of my research findings.

\section{Key findings}
\label{eval:findings}


The main finding is that it is possible to build a Haskell eDSL as a library that allows for declarative spreadsheet construction.
A user specifies the connections between the spreadsheet elements, and the library functions produce the elements layout automatically.
Despite being simple, the language requires the knowledge of Haskell at least at the beginner level.
That is why, the target audience of the language is Haskell programmers.
They are expected to generate the spreadsheets and pass these spreadsheets to non-programmers.

Another finding is that such an eDSL provides the means to construct statically typed formulas.
This feature allows for typechecking the formula expressions at compile time.
Such checks prevent runtime errors, such as invalid formula arguments, in spreadsheet applications.

The third finding is that it is possible to use the existing Haskell features in the programs written in the new eDSL.
First, formulas can be curried, composed, and applied to ranges of values and cell references.
Second, algebraic data types can be used to model the value domains.
Third, the \texttt{do-notation} works because the eDSL has a monadic interface.

\section{Results and findings}
\label{eval:finding-results}

First, the result of my research is the \texttt{clerk} Haskell package.
It is publicly available on GitHub \cite{danko_clerk_2023} and Hackage \cite{hackage_clerk_2023}.
Haskell programmers may discover the package there, read through the examples from the package description and use \texttt{clerk} in their programs.
The examples are based on simple problems and demonstrate most features of the eDSL.

Second, the type checking of formulas works at compile time.
A user first specifies the formula signature - a name and the types of arguments.
Later, the user can apply a formula to arguments.

Third, the examples from the package description demonstrate how to apply certain Haskell features to write type-safe monadic code using \texttt{clerk}.

\section{Previous research}
\label{eval:previous-research}

Currently, the major spreadsheet applications provide limited tools for declarative type-safe construction of spreadsheets.

\subsection{User-defined data types}
\label{subsec:user-defined-data-types}
Firstly, Microsoft Excel provides creation of user-defined data types \cite{excel_custom_types}.
A user may group the columns of values into records that represent user-defined types.
Additionally, they may use a user-defined data type as a type of a field of another user-defined data type.

In contrast, \texttt{clerk} allows to type-safely group values into Haskell records.
This approach allows for building type-safe composite records before importing them into a spreadsheet application.

\subsection{User-defined functions}

Microsoft Excel introduced the \texttt{LAMBDA} function \cite{excel_lambda} that allows for user-defined functions.
Google Sheets also provides the \texttt{LAMBDA} function \cite{sheets_lambda} that should immediately be applied to a value.
Thus, in this section, I focus on the \texttt{LAMBDA} function from Microsoft Excel.

First, the \texttt{LAMBDA} function can be recorded and shared between spreadsheets.
Second, this function allows for recursion and usage of other functions, including user-defined ones.
Third, user-defined functions are dynamically-typed.
The argument types can be specified as comments to a recorded user-defined function.

In comparison to \texttt{LAMBDA}, \texttt{clerk} allows a user to declare functions in Haskell.
First, these functions can be imported into other Haskell modules or programs.
Second, these functions may represent compositions of functions from the target spreadsheet system, including user-defined ones.
Third, these functions are statically typed. Their arguments may be documented using Haddock.

\subsection{Declarative layout}

Currently, Microsoft Excel has support for automatic layout of data upon importing it \cite{excel_custom_types}.
However, layout customization still should be done manually.

There are libraries for several languages that derive the layout from user specification of spreadsheet elements and the connections between them.
The \texttt{poi} library for Python can write spreadsheets based on user data \cite{wang_poi_nodate}.
To my knowledge, the library does not allow to specify the connections between new eleemnts and previously built elements.



\section{Discrepancies}
\label{eval:discrepancies}

\section{Limitations}
\label{eval:limitations}

\section{Unexpected findings}
\label{eval:unexpected}

\section{Outer applications}
\label{eval:outer-applications}



% \begin{longtable}{c|c}
% \caption[This is the title I want to appear in the List of Tables]{Simulation Parameters} \label{table:fifsimulation_params} \\
% \hline
% A & B  \\
% \hline
% \endfirsthead
% \multicolumn{2}{@{}l}{} \\
% \hline
% A & B \\
% \hline
% \endhead
% \hline
%  \textbf{Parameter} & \textbf{Value}\\
%  \hline
%  Number of vehicles & $|\mathcal{V}|$\\
%  \hline
%  Number of RSUs & $|\mathcal{U}|$\\
%  \hline
%  RSU coverage radius & 150 m\\
%  \hline
%  V2V communication radius & 30 m\\
%  \hline
%  Smart vehicle antenna height & 1.5 m\\
%  \hline
%  RSU antenna height & 25 m\\
%  \hline
%  Smart vehicle maximum speed & $v_{max}$ m/s\\
%  \hline
%  Smart vehicle minimum speed & $v_{min}$ m/s\\
%  \hline
%  Common smart vehicle cache capacities & $[50, 100, 150, 200, 250]$ mb\\
%  \hline
%  Common RSU cache capacities & $[5000,1000,1500,2000,2500]$ mb\\
%  \hline
%  Common backhaul rates & $[75, 100, 150]$ mb/s\\
%  \hline
% \end{longtable}

% \begin{figure}[hbt]
% \centering
% \includegraphics[]{figs/inno.png}
% \caption{One kernel at $x_s$ (\emph{dotted kernel}) or two kernels at
% $x_i$ and $x_j$ (\textit{left and right}) lead to the same summed estimate
% at $x_s$. This shows a figure consisting of different types of
% lines. Elements of the figure described in the caption should be set in
% italics, in parentheses, as shown in this sample caption.}
% \label{fig:fifex}
% \end{figure}

\ldots
