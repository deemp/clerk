% \documentclass[oneside]{report}
\documentclass[oneside,final,14pt,a4paper]{extreport}
% \documentclass[journal,onecolumn,a4paper,12pt]{IEEEtran}
\usepackage[T2A]{fontenc}


\usepackage{vmargin}
\setpapersize{A4}
\setmarginsrb{2.5cm}{2cm}{2cm}{2cm}{0pt}{10mm}{0pt}{13mm}
\usepackage{setspace}
\sloppy
\setstretch{1.5}
\usepackage{indentfirst}
\parindent=1.25cm

%%%%% ADDED TO SUPPORT TT BOLD FACES %%%%
\DeclareFontShape{OT1}{cmtt}{bx}{n}{<5><6><7><8><9><10><10.95><12><14.4><17.28><20.74><24.88>cmttb10}{}
\renewcommand{\ttdefault}{pcr}
%%%%% END %%%%%%%%%%%%%%%%%%%%%%%%%%%%%%% 
\usepackage{atbegshi,picture}
\AtBeginShipout{\AtBeginShipoutUpperLeft{%
  \put(\dimexpr\paperwidth-1cm\relax,-1.5cm){\makebox[0pt][r]{\includegraphics[width=3cm]{figs/inno}}}%
}}


\usepackage[english]{babel}
\usepackage[backend=biber,style=ieee,autocite=inline]{biblatex}
\bibliography{ref.bib}
\DefineBibliographyStrings{english}{%
  bibliography = {References},}
\usepackage{blindtext}
\usepackage{pdfpages}
\newenvironment{bottompar}{\par\vspace*{\fill}}{\clearpage}
\usepackage{amsmath,amsfonts}
\usepackage{url}

\usepackage{amsthm}
\newtheorem{theorem}{Theorem}
\newtheorem{corollary}{Corollary}
\newtheorem{lemma}{Lemma}
\newtheorem{proposition}{Proposition}
\theoremstyle{definition}
\newtheorem{definition}{Definition}
\theoremstyle{remark}
\newtheorem*{remark}{Remark}
\theoremstyle{remark}
\newtheorem*{example}{Example}


\usepackage{titlesec}
\usepackage{float}
\usepackage{graphicx}
\graphicspath{{figs/}} %path to images
\usepackage{array}
\usepackage{multirow,array}
\usepackage{caption}
\usepackage{subcaption}
\usepackage{hyperref}
\usepackage{paralist}
\usepackage{listings}
\usepackage{zed-csp}
\usepackage{fancyhdr}
\usepackage{csquotes}
\usepackage{color}
\usepackage{anyfontsize}
\usepackage{mathptmx}
\usepackage{t1enc}
\usepackage{syntax}
\usepackage{chngcntr}
\usepackage{upgreek} 
\usepackage{bm}
% \usepackage{hyperref}
\usepackage{multicol}
\usepackage{setspace}
\usepackage{booktabs}
\usepackage{multirow}
\usepackage{longtable}
\usepackage[font=singlespacing, labelfont=bf]{caption}
\counterwithout{table}{chapter}
\renewcommand{\thetable}{\Roman{table}}
%Hints
\newcommand\pic[1]{(Fig. \ref{#1})} %Ref on figure
\newcommand\tab[1]{(Tab. \ref{#1})} %Ref on table

\setlength{\headheight}{32.0976pt}
\usepackage{enumitem}
\newlist{inlinelist}{enumerate*}{1}
\setlist*[inlinelist,1]{%
  label=(\arabic*),
}

\usepackage{cleveref}
\crefname{table}{table}{tables}
\Crefname{table}{Table}{Tables}
\crefname{figure}{Fig.}{figures}
\Crefname{figure}{Fig.}{Figures}
\crefname{listing}{List.}{List.}
\Crefname{listing}{List.}{List.}

% --- START --- minted

% https://github.com/latextemplates/IEEE/blob/main/paper-conference-minted.tex
\usepackage[newfloat]{minted}
\setminted{
  % Add line numbers, can be added per listing 
  % linenos=true,
  % Line numbers not flowing out of the margin
  numbersep=12pt,
  xleftmargin=19pt,
  breaklines=true,
  breakbytoken=true,
  % For the very long tokens, use
  % "breakanywhere, breakbytokenanywhere=false"
  % on the listing level. Otherwise breakanywhere will not have any effect.
  breakbytokenanywhere=true
}
\renewcommand{\theFancyVerbLine}{\textcolor{gray}{\arabic{FancyVerbLine}}}

\captionsetup[listing]{labelfont={normalfont}, name={List.}, labelsep=period}
\counterwithout{listing}{chapter}

\newcommand{\hs}{\mintinline[fontsize=\normalsize]{haskell}}

\newcommand{\clerk}{\texttt{clerk} }

% --- END --- minted

\setcounter{secnumdepth}{4}
\captionsetup[table]{labelfont={normalfont}, name={TABLE}, labelsep={newline}}
\counterwithout{table}{chapter}
\renewcommand{\thetable}{\Roman{table}}
\setlength{\parindent}{2em} 
\DeclareCaptionLabelSeparator{figSep}{.\quad}
\captionsetup[figure]{labelfont={normalfont}, name={Fig.}, labelsep=period}
\counterwithout{figure}{chapter}

\titleformat{\section}[hang]{\fontsize{20}{24}\selectfont\filcenter}{\Roman{section}}{1em}{}
\titleformat{\subsection}[hang]{\itshape}{\Alph{subsection}.}{1em}{}[]
\titleformat{\subsubsection}[runin]{\itshape}{\arabic{subsubsection})}{1em}{}[$:$]
\titlespacing{\subsubsection}{1em}{1em}{1em}
\titleformat{\paragraph}[runin]{\itshape}{\alph{paragraph})}{1em}{}[$:$\quad]
\titlespacing{\paragraph}{2em}{1em}{1em}

\pagestyle{fancyplain}

% remember section title
\renewcommand{\chaptermark}[1]%
	{\markboth{\chaptername~\thechapter~--~#1}{}}

% subsection number and title
\renewcommand{\sectionmark}[1]%
	{\markright{\thesection\ #1}}

\rhead[\fancyplain{}{\bf\leftmark}]%
      {\fancyplain{}{\bf\thepage}}
\lhead[\fancyplain{}{\bf\thepage}]%
      {\fancyplain{}{\bf\rightmark}}
\cfoot{} %bfseries


\newcommand{\dedication}[1]
   {\thispagestyle{empty}
     
   \begin{flushleft}\raggedleft #1\end{flushleft}
}

\begin{document}

%\includepdf[pages=-]{title.pdf}
%\tableofcontents
%\listoftables
%\listoffigures


\newpage
%\begin{abstract}
abstract \ldots
\end{abstract}
% Depend on above part
\setcounter{page}{1}
% \chapter{Введение}
\label{chap:intro}
Мяу \cite{A,B,C}
%\chapter{Literature Review}
\label{chap:lr}
\chaptermark{Second Chapter Heading}


\Blindtext[2]

\section{Another Section}
\Blindtext[1]

% \chapter{Методология}
\label{chap:met}
% \include{chapters/chap}
\include{chapters/chapter4/example1.tex}
% D

% module Chapters.Chapter4.Example2 where

% E

\section{Example 2. Multiplication Table}
\label{sec:ex2}

The \clerk library supported connections between cells, tables, and sheets via reference offsets. In other words, the library allowed to set the relative positions of objects in a spreadsheet.

In this section, I described a spreadsheet with a multiplication table to demonstrate the connections between tables within a sheet.

The \cref{example2:tableValues} shows the resulting multiplication table and the \cref{example2:tableFormulas} shows the formulas inside that table.

\begin{figure}[h]
  \centering
  \includegraphics[scale=0.3]{demoValues.png}
  \caption{Multiplication table}
  \label{example2:tableValues}
\end{figure}

\begin{figure}[h]
  \centering
  \includegraphics[scale=0.3]{demoFormulas.png}
  \caption{Multiplication table with formulas}
  \label{example2:tableFormulas}
\end{figure}

\subsection{Language Extensions}

First, I enabled the extension that simplified the work with string literals.

\begin{listing}[!h]
  \begin{minted}{haskell}
  {-# LANGUAGE OverloadedStrings #-}
\end{minted}
  \caption{Language extensions}
  \label{example2:languageExtensions}
\end{listing}

\subsection{Imports}

Then, I imported the necessary modules.
Here, I included \hs{Control.Monad} as the task required performing repetitive monadic actions.
The module \hs{Data.Text} was used for constructing sheet names.
The \hs{Lens.Micro} module provided optics for working with cell references.

\begin{listing}[!h]
  \begin{minted}{haskell}
  import Clerk
  import Control.Monad (forM, forM_, void)
  import qualified Data.Text as T
  import Lens.Micro ((&), (+~), (^.))
\end{minted}
  \caption{Imports}
  \label{example2:imports}
\end{listing}

\subsection{Tables}

\newcommand{\vh}{Vertical Header}
\newcommand{\hh}{Horizontal Header}

In my case, a table was a logically grouped set of cell references.
Each table should contain at least a single cell.
The tables that I constructed were the following:

\begin{itemize}
  \item A column with row numbers ("\vh");
  \item A row with column numbers ("\hh");
  \item Tables with results of multiplication of the numbers from these headers ("Inner Table").
\end{itemize}

\subsubsection{\vh}
\label{example2:verticalHeaderSection}

I planned to build a \vh as in \cref{example2:verticalHeader}.

\begin{figure}[h]
  \centering
  \includegraphics[scale=0.3]{vertical.png}
  \caption{\vh}
  \label{example2:verticalHeader}
\end{figure}

To achieve this goal, I had used the \clerk monadic interface to constructing rows and setting their positions within a sheet.

One of the monads provided by \clerk was the \hs{RowI} monad. Its name meant that inside that monad, I had to use a projection function from an \hs{input} type to \hs{CellData} type. The \hs{CellData} type was the type of spreadsheet values.

The effect of the \hs{RowI} monad was writing a template of a horizontal block of cells. The values obtained from \hs{input} via the projection function were placed onto a sheet according to that template.

I used the monadic function \hs{columnF} for constructing a \hs{RowI}. This function:
\begin{itemize}
  \item added a projected value into the template;
  \item added the formatting to the template;
  \item shifted a pointer to the next free cell in a template.
\end{itemize}

A vertical header contained just a single cell in each its row (\cref{example2:verticalHeader}). Such a header could be represented as several rows with verticall offset equal to \hs{1}. So, the rows were below each other.

I used a \hs{RowI} with one integer as an \hs{input} for a single row cell. This input was the index of the row within the \vh. I set the cell formatting to \hs{blank} since I did not need styling. I placed the rows for each input value onto a \hs{Sheet} and collected the references.

The placement of rows depended on input reference \hs{ref}. I could shift the rows within a sheet by varying this reference. Additionally, I decided to shift all rows vertically by \hs{2} from the input reference.

That said, the function \hs{mkVertical} (\cref{example2:verticalHeaderCode}) for each number from a given range of numbers (size of the multiplication table) and for each index of such a number runs a monadic function in the \hs{Sheet} monad via \hs{forM}. This function \hs{placeIn} at \hs{ref} shifted by \hs{index + 2} vertically places a number according to a \hs{RowI} produced by \hs{columnF}. \hs{columnF} makes a blank cell and uses a constant function as a projection function from the \hs{input} type of \hs{RowI} which is \hs{Int} in this case.

\begin{listing}[!h]
  \begin{minted}{haskell}
mkVertical :: Ref () -> [Int] -> Sheet [Ref Int]
mkVertical ref numbers =
  forM (zip [0 ..] numbers) $ \(idx, number) ->
    placeIn
    (ref & row +~ idx + 2)
    number
    ((columnF blank (const number)) :: RowI Int (Ref Int))
\end{minted}
  \caption{\vh}
  \label{example2:verticalHeaderCode}
\end{listing}

\subsubsection{\hh}
\label{example2:horizontalHeaderSection}

Next, I added a \hh (\Cref{example2:horizontalHeader}).

\begin{figure}[h]
  \centering
  \includegraphics[scale=0.3]{horizontal.png}
  \caption{\hh}
  \label{example2:horizontalHeader}
\end{figure}

I made a row of numbers and collected the references to all its cells (\cref{example2:horizontalHeaderCode}). As the type of inputs was not important, I used the \hs{Row} type. This type assumed that the \hs{input} type was unit, or \hs{()}. In the \hs{Sheet} monad, I placed this row starting at a cell that was shifted horizontally by \hs{2} from the input reference. In this case, I used \hs{forM} to run the monadic function \hs{columnF} several times in the \hs{Row} monad.

\begin{listing}[!h]
  \begin{minted}{haskell}
mkHorizontal :: Ref () -> [Int] -> Sheet [Ref Int]
mkHorizontal ref numbers =
  place
  (ref & col +~ 2)
  ((forM numbers $ \n -> columnF blank (const n)) :: Row [Ref Int])
\end{minted}
  \caption{\hh}
  \label{example2:horizontalHeaderCode}
\end{listing}

\subsubsection{Inner Table}

The next step was to construct a table (\cref{example2:table}).

\begin{figure}[h]
  \centering
  \includegraphics[scale=0.25]{table.png}
  \caption{Inner Table}
  \label{example2:table}
\end{figure}

The function \hs{mkTable} (\cref{example2:tableCode}) took a list of references to cells from the \vh and the \hh. For each such pair of references, it monadically made a coordinate via \hs{mkCoords}. This way, the \hs{mkCoords} function could access the current state of a \hs{Sheet}, e.g., its name and the workbook file path.

I decided to use single-cell rows for each Inner Table cell. This way, I could better align the positions of Inner Table cells with those of cells from headers. Moreover, I used the \hs{Row ()} type since I did not plan to use the references to table cells in other places.

The \hs{columnF} function used a projection to the formula \hs{r .* c}.
In other words, the reference to a cell from the \vh multiplied by the reference to a cell from the \hh.

\begin{listing}[!h]
  \begin{minted}{haskell}
mkTable :: [(Ref Int, Ref Int)] -> Sheet ()
mkTable cs =
  forM_ cs $ \(r, c) -> do
    coords <- mkCoords (c ^. col) (r ^. row)
    place coords ((columnF_ blank (const (r .* c))) :: Row ())  
\end{minted}
  \caption{Inner Table}
  \label{example2:tableCode}
\end{listing}

\subsection{Sheet}

Finally, I produced a complete \hs{Sheet ()}.
Here, I set the initial reference at \hs{B2}.
Then, I produced a list of numbers and constructed the headers.
Lastly, I used a list comprehension to generate all pairs of references to header cells and run \hs{mkTable} on this list of pairs.

\begin{listing}[!h]
  \begin{minted}{haskell}
sheet :: Sheet ()
sheet = do
  start <- mkRef' @"B2"
  let numbers = [1 .. 9]
  cs <- mkHorizontal start numbers
  rs <- mkVertical start numbers
  mkTable [(r, c) | r <- rs, c <- cs]
\end{minted}
  \caption{Inner Table}
  \label{example2:sheetCode}
\end{listing}

\subsection{Result}

To observe the resulting sheet, I run the \hs{main} function from \cref{example2:mainCode}.

\begin{listing}[!h]
  \begin{minted}{haskell}
main :: IO ()
main = writeXlsx "example2.xlsx" [(T.pack "List 1", void sheet)]
\end{minted}
  \caption{Writing result}
  \label{example2:mainCode}
\end{listing}

The resulting table looked as in \cref{example2:table}.
\section{Example 3. Volume and Pressure table}

Sometimes, a user may want to style her generated spreadsheet.
This example shows how to model a problem domain and style the generated spreadsheet.

\subsection{The goal}

The goal was to generate a spreadsheet that calculated the pressure data given volume data and several constants.

The final spreadsheet had the form as in \Cref{fig:volumePressure}.

\begin{figure}[h]
  \centering
  \includegraphics[scale=0.3]{Chapter4/volumePressure.png}
  \caption{Volume and pressure table}
  \label{fig:volumePressure}
\end{figure}

With formulas enabled, I obtained a spreadsheet as in \Cref{fig:volumePressureFormulas}.

\begin{figure}[h]
  \centering
  \includegraphics[scale=0.3]{Chapter4/volumePressureFormulas.png}
  \caption{Volume and pressure table with formulas enabled}
  \label{fig:volumePressureFormulas}
\end{figure}

\subsection{Language Extensions}

First, I enabled several language extensions.
\begin{itemize}
  \item \hs{DataKinds} allowed to promote values to type level.
  \item \hs{DuplicateRecordFields} allowed to use the same field names in multiple record data types.
  \item \hs{ImportQualifiedPost} allowed to use module imports of a specific form.
  \item \hs{OverloadedRecordDot} allowed to use the \hs{.} operator for accessing the record fields.
  \item \hs{RecordWildCards} allowed to omit the field names in records in certain cases.
\end{itemize}

\begin{listing}[!h]
  \begin{minted}{haskell}
{-# LANGUAGE DataKinds #-}
{-# LANGUAGE DuplicateRecordFields #-}
{-# LANGUAGE ImportQualifiedPost #-}
{-# LANGUAGE OverloadedRecordDot #-}
{-# LANGUAGE RecordWildCards #-}
{-# LANGUAGE TypeApplications #-}
\end{minted}
\caption{Language extensions}
\label{example3:extensions}
\end{listing}

\subsection{Imports}

Next, I imported the necessary Modules.

\begin{listing}[!h]
  \begin{minted}{haskell}
import Clerk
import Codec.Xlsx qualified as X
import Codec.Xlsx.Formatted qualified as X
import Lens.Micro ((%~), (&), (+~), (?~))
\end{minted}
\caption{Imports}
\label{example3:imports}
\end{listing}

\subsection{Tables}

The tables that I was going to construct were:

- A table per a constant's value (three of them)
- A volume and pressure table
- A constants' header
- A volume and pressure header

\subsubsection{Constants values}

\begin{figure}[h]
  \centering
  \includegraphics[scale=0.3]{Chapter4/constants.png}
  \caption{Constants values table}
  \label{fig:constants}
\end{figure}

In my case, each constant had the same type of the numeric value - `Double`.
That is why, I constructed a table with a single row per a constant and later placed the constants' tables under each other. I stored constant data in a record.

\begin{listing}[!h]
  \begin{minted}{haskell}
data ConstantData a = ConstantData
  { constantName :: String
  , constantSymbol :: String
  , constantValue :: a
  , constantUnits :: String
  }
\end{minted}
\caption{Constant data}
\label{example3:constantData}
\end{listing}

Next, I grouped the constants into a record. The parameter \texttt{f} may be an arbitrary type of kind \texttt{* -> *}.

\begin{listing}[!h]
  \begin{minted}{haskell}
data Constants f = Constants
  { gasConstant :: f Double
  , numberOfMoles :: f Double
  , temperature :: f Double
  }

type ConstantsInput = Constants ConstantData
\end{minted}
\caption{Constants}
\label{example3:constants}
\end{listing}

Following that, I recorded the constants data.

\begin{listing}[!h]
  \begin{minted}{haskell}
constants :: ConstantsInput
constants =
  Constants
    { gasConstant =
        ConstantData "GAS CONSTANT" "R" 0.08206 "L.atm/mol.K"
    , numberOfMoles =
        ConstantData "NUMBER OF MOLES" "n" 1 "moles"
    , temperature =
        ConstantData "TEMPERATURE(K)" "T" 273.2 "K"
    }
\end{minted}
\caption{Constant values}
\label{example3:constantValues}
\end{listing}

Now, I made a \texttt{RowI} for a constant input.
I used a \texttt{RowI} because this row requires the input type.
I would later use this row for each constant separately.

I got a pair of outputs:

- Top left cell of a constant table. That is, the cell with that constant name.
- The value of the constant.

Later, I would use these outputs to relate the positions of tables on a sheet.

Here, I used \texttt{lightBlue} from the \Cref{sec:styles}.

\begin{listing}[!h]
  \begin{minted}{haskell}
constant :: ToCellData a => RowI (ConstantData a) (Ref (), Ref a)
constant = do
  refTopLeft <- columnF lightBlue constantName
  columnF_ lightBlue constantSymbol
  refValue <- columnF (lightBlue .& with2decimalDigits) constantValue
  columnF_ lightBlue constantUnits
  return (refTopLeft, refValue)
\end{minted}
\caption{Language extensions}
\label{example3:constantCode}
\end{listing}

\subsubsection{Volume and pressure values}

\begin{figure}[h]
  \centering
  \includegraphics[scale=0.3]{Chapter4/valuesFormulas.png}
  \caption{Volume and pressure table with formulas enabled}
  \label{fig:valuesFormulas}
\end{figure}

I used the data and combined it with the constants to fill table \Cref{fig:valuesFormulas}

\begin{listing}[!h]
  \begin{minted}{haskell}
newtype Volume = Volume {volume :: Double}

volumeData :: [Volume]
volumeData = Volume <$> [1 .. 10]
\end{minted}
\caption{Language extensions}
\label{example3:volumeData}
\end{listing}

I made a helper type to pass the constants references in a structured way.

\begin{listing}[!h]
  \begin{minted}{haskell}
type ConstantsRefs = Constants Ref
\end{minted}
\caption{Constant references}
\label{example3:constantRefs}
\end{listing}

Next, I defined a function to produce a row for volume and pressure.

\begin{listing}[!h]
  \begin{minted}{haskell}
values :: ConstantsRefs -> RowI Volume ()
values Constants{..} = do
  refVolume <- columnF alternatingColors volume
  let pressure' = gasConstant .* numberOfMoles .* temperature ./ refVolume
  columnF_ (alternatingColors .& with2decimalDigits) (const pressure')
\end{minted}
\caption{Values}
\label{example3:valuesCode}
\end{listing}

\subsubsection{Constants' header}

\begin{figure}[h]
  \centering
  \includegraphics[scale=0.3]{Chapter4/constantsHeader.png}
  \caption{Constants header}
  \label{fig:constantsHeader}
\end{figure}

I did not use records here. Instead, I put the names of the columns straight into the `Row`. The outputs were the coordinates of the top left cell and the top right cell of this table.

\begin{listing}[!h]
  \begin{minted}{haskell}
constantsHeader :: Row (Ref (), Ref ())
constantsHeader = do
  let style :: FormatCell
      style = blue .& alignedCenter
  refTopLeft <- columnWF 20 style (const "constant")
  columnWF_ 8 style (const "symbol")
  columnF_ style (const "value")
  refTopRight <- columnWF 13 style (const "units")
  return (refTopLeft, refTopRight)
\end{minted}
\caption{Language extensions}
\label{example3:extensions}
\end{listing}

\subsubsection{Volume \& Pressure header}

\begin{figure}[h]
  \centering
  \includegraphics[scale=0.3]{Chapter4/valuesHeader.png}
  \caption{Header for the volume and pressure table}
  \label{fig:valuesHeader}
\end{figure}

For this header, I also put the names of columns straight into a row.

\begin{listing}[!h]
  \begin{minted}{haskell}
valuesHeader :: Row (Ref ())
valuesHeader = do
  refTopLeft <- columnWF 12 green (const "VOLUME (L)")
  columnWF_ 16 green (const "PRESSURE (atm)")
  return refTopLeft
\end{minted}
\caption{Language extensions}
\label{example3:extensions}
\end{listing}

\subsection{Sheet builder}

Finally, I combined all rows.

\begin{listing}[!h]
  \begin{minted}{haskell}
sheet :: Sheet ()
sheet = do
  start <- mkCoords 2 2
  (constantsHeaderTL, constantsHeaderTopRight) <- place start constantsHeader
  (gasTopLeft, gas) <- placeIn (constantsHeaderTL & row +~ 2) constants.gasConstant constant
  (nMolesTopLeft, nMoles) <- placeIn (gasTopLeft & row +~ 1) constants.numberOfMoles constant
  temperature <- snd <$> placeIn (nMolesTopLeft & row +~ 1) constants.temperature constant
  valuesHeaderTopLeft <- place (constantsHeaderTopRight & col +~ 2) valuesHeader
  placeIns (valuesHeaderTopLeft & row +~ 2) volumeData (values $ Constants gas nMoles temperature)
\end{minted}
\caption{Language extensions}
\label{example3:extensions}
\end{listing}

\subsection{Styles}
\label{sec:styles}

Below, I list the expressions used for sheet styling.

\begin{listing}[!h]
  \begin{minted}{haskell}
blue, lightBlue, green, lightGreen :: FormatCell
blue = mkColor (hex @"#FF99CCFF")
lightBlue = mkColor (hex @"#90CCFFFF")
green = mkColor (hex @"#FF00FF00")
lightGreen = mkColor (hex @"#90CCFFCC")

alternatingColors :: FormatCell
alternatingColors index = (if even index then lightGreen else lightBlue) index
\end{minted}
\caption{Language extensions}
\label{example3:extensions}
\end{listing}

Additionally, I composed an \texttt{FCTransform} for the number format.
Such a transform was used to accumulate cell formatting.

\begin{listing}[!h]
  \begin{minted}{haskell}
with2decimalDigits :: FCTransform
with2decimalDigits fcTransform =
  fcTransform & X.formattedFormat %~ X.formatNumberFormat ?~ X.StdNumberFormat X.Nf2Decimal
\end{minted}
\caption{Language extensions}
\label{example3:extensions}
\end{listing}

Finally, I made a transform for centering the cell contents.

\begin{listing}[!h]
  \begin{minted}{haskell}
alignedCenter :: FCTransform
alignedCenter = horizontalAlignment X.CellHorizontalAlignmentCenter
\end{minted}
\caption{Language extensions}
\label{example3:extensions}
\end{listing}

% \include{chapters/chapter4-3.tex}
% \include{chapters/chapter4-cont}
% \include{chapters/chap}
% \chapter{Evaluation and Discussion}
\label{chap:eval}

This chapter analyzes the research results.

\Cref{eval:findings} presents the main findings that are connected with the research purpose.
\Cref{eval:finding-results} interprets how the research results support these findings.
\Cref{eval:previous-research} contrasts my findings with the results of the past researches.
\Cref{eval:discrepancies} explains the discrepancies between my results and findings.
\Cref{eval:limitations} describes the limitations of my research.
\Cref{eval:unexpected} lists the unexpected findings.
\Cref{eval:outer-applications} suggests the possible applications of my research findings.

\section{Key findings}
\label{eval:findings}


The main finding is that it is possible to build a Haskell eDSL as a library that allows for declarative spreadsheet construction.
A user specifies the connections between the spreadsheet elements, and the library functions produce the elements layout automatically.
Despite being simple, the language requires the knowledge of Haskell at least at the beginner level.
That is why, the target audience of the language is Haskell programmers.
They are expected to generate the spreadsheets and pass these spreadsheets to non-programmers.

Another finding is that such an eDSL provides the means to construct statically typed formulas.
This feature allows for typechecking the formula expressions at compile time.
Such checks prevent runtime errors, such as invalid formula arguments, in spreadsheet applications.

The third finding is that it is possible to use the existing Haskell features in the programs written in the new eDSL.
First, formulas can be curried, composed, and applied to ranges of values and cell references.
Second, algebraic data types can be used to model the value domains.
Third, the \texttt{do-notation} works because the eDSL has a monadic interface.

\section{Results and findings}
\label{eval:finding-results}

First, the result of my research is the \texttt{clerk} Haskell package.
It is publicly available on GitHub \cite{danko_clerk_2023} and Hackage \cite{hackage_clerk_2023}.
Haskell programmers may discover the package there, read through the examples from the package description and use \texttt{clerk} in their programs.
The examples are based on simple problems and demonstrate most features of the eDSL.

Second, the type checking of formulas works at compile time.
A user first specifies the formula signature - a name and the types of arguments.
Later, the user can apply a formula to arguments.

Third, the examples from the package description demonstrate how to apply certain Haskell features to write type-safe monadic code using \texttt{clerk}.

\section{Previous research}
\label{eval:previous-research}

Currently, the major spreadsheet applications provide limited tools for declarative type-safe construction of spreadsheets.

\subsection{User-defined data types}
\label{subsec:user-defined-data-types}
Firstly, Microsoft Excel provides creation of user-defined data types \cite{excel_custom_types}.
A user may group the columns of values into records that represent user-defined types.
Additionally, they may use a user-defined data type as a type of a field of another user-defined data type.

In contrast, \texttt{clerk} allows to type-safely group values into Haskell records.
This approach allows for building type-safe composite records before importing them into a spreadsheet application.

\subsection{User-defined functions}

Microsoft Excel introduced the \texttt{LAMBDA} function \cite{excel_lambda} that allows for user-defined functions.
Google Sheets also provides the \texttt{LAMBDA} function \cite{sheets_lambda} that should immediately be applied to a value.
Thus, in this section, I focus on the \texttt{LAMBDA} function from Microsoft Excel.

First, the \texttt{LAMBDA} function can be recorded and shared between spreadsheets.
Second, this function allows for recursion and usage of other functions, including user-defined ones.
Third, user-defined functions are dynamically-typed.
The argument types can be specified as comments to a recorded user-defined function.

In comparison to \texttt{LAMBDA}, \texttt{clerk} allows a user to declare functions in Haskell.
First, these functions can be imported into other Haskell modules or programs.
Second, these functions may represent compositions of functions from the target spreadsheet system, including user-defined ones.
Third, these functions are statically typed. Their arguments may be documented using Haddock.

\subsection{Declarative layout}

Currently, Microsoft Excel has support for automatic layout of data upon importing it \cite{excel_custom_types}.
However, layout customization still should be done manually.

The \texttt{poi} library for Python can write spreadsheets based on user data \cite{wang_poi_nodate}.
The library seems to not allow to specify the connections between new elements and previously built elements.

Such functionality, on the other hand, is present in the Kotlin library Geschikt \cite{noauthor_sikrinickgeshikt_nodate}.
The library provides a declarative API to building Google Sheets.
A user may create an cell object and then access its reference in the expressions that follow.
The downside is that there seems to be no way to use a reference outside of a parent element.

In \texttt{clerk}, it is possible to get a reference to a specific cell.
It is the user's responsibility to return that reference from the function where the cell is created.
The cell references can be

\section{Discrepancies}
\label{eval:discrepancies}

\section{Limitations}
\label{eval:limitations}

First of all, currently, the eDSL is not usable by non-programmers.
Perhaps the implementation of a graphical user interface to the eDSL may solve this problem.

Next, another problem is that the library generates spreadsheets using the limited number of formatting tools provided by the underlying \texttt{xlsx} library.
Thus, if a user edits a generated part of a spreadsheet, these edits will be removed upon the next spreadsheet generation.
This limitation can be overcome by copying just the data from the generated parts into new sheets and applying formatting there.

\section{Unexpected findings}
\label{eval:unexpected}

\section{Outer applications}
\label{eval:outer-applications}

The \texttt{clerk} library allows to produce correct by construction spreadsheets.
I suppose that this property can make the library useful in real world systems.

First, this library can be used as a part of data pipelines.
The program using the \texttt{clerk} library can generate data-transforming spreadsheets.
There will be regions for input data, output data, and formulas.
The formulas will act onto input data to produce output data.
Other programs may write to or read the outputs from such data-transforming spreadsheets.

Second, such a library may be useful in areas that require correct computations.
As far as I know from Haskell community chats, some financial and biotechnological organizations rely on Haskell.
Moreover, some people use Excel for recording the results of scientific experiments.

Third, it is possible to use \texttt{clerk} for generating reports.
The library provides basic styling capabilities.
When the library is used with a Haskell repl, the resulting spreadsheets can be generated rapidly.
Thus, the spreadsheet editor may quickly observe the changes.

Fourth, the library may be used by pixel art fans.
It is quite trivial to fill the table cells with colors of pixels of an image.

Fifth, the library can be used for teaching type-level programming in Haskell.
More specifically, the library includes modules that provide type-level parsing capabilities.
These modules enable compile-time cell address and color values checks.

% \begin{longtable}{c|c}
% \caption[This is the title I want to appear in the List of Tables]{Simulation Parameters} \label{table:fifsimulation_params} \\
% \hline
% A & B  \\
% \hline
% \endfirsthead
% \multicolumn{2}{@{}l}{} \\
% \hline
% A & B \\
% \hline
% \endhead
% \hline
%  \textbf{Parameter} & \textbf{Value}\\
%  \hline
%  Number of vehicles & $|\mathcal{V}|$\\
%  \hline
%  Number of RSUs & $|\mathcal{U}|$\\
%  \hline
%  RSU coverage radius & 150 m\\
%  \hline
%  V2V communication radius & 30 m\\
%  \hline
%  Smart vehicle antenna height & 1.5 m\\
%  \hline
%  RSU antenna height & 25 m\\
%  \hline
%  Smart vehicle maximum speed & $v_{max}$ m/s\\
%  \hline
%  Smart vehicle minimum speed & $v_{min}$ m/s\\
%  \hline
%  Common smart vehicle cache capacities & $[50, 100, 150, 200, 250]$ mb\\
%  \hline
%  Common RSU cache capacities & $[5000,1000,1500,2000,2500]$ mb\\
%  \hline
%  Common backhaul rates & $[75, 100, 150]$ mb/s\\
%  \hline
% \end{longtable}

% \begin{figure}[hbt]
% \centering
% \includegraphics[]{figs/inno.png}
% \caption{One kernel at $x_s$ (\emph{dotted kernel}) or two kernels at
% $x_i$ and $x_j$ (\textit{left and right}) lead to the same summed estimate
% at $x_s$. This shows a figure consisting of different types of
% lines. Elements of the figure described in the caption should be set in
% italics, in parentheses, as shown in this sample caption.}
% \label{fig:fifex}
% \end{figure}

\ldots

%\chapter{Conclusion}
\label{chap:conclusion}


\ldots



%% REFERENCES
\printbibliography[heading=bibintoc,title={Bibliography cited}]
% \appendix
\chapter{Extra Stuff}
\blindtext

\chapter{Even More Extra Stuff}
\blindtext
\end{document}

